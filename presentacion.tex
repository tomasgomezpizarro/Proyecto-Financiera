\documentclass{beamer}
\usetheme{Madrid}  % Replaced UniKlu (missing) with a standard built-in theme
\usepackage{booktabs}
\usepackage{amssymb}

\title{Diferencias finitas para valorar opciones americanas\\con cambio de variable $Z \to \ln S$}
 	

\author{Mikael Rodriguez \\ Tomás Gómez Pizarro}
\institute{FAMAF - Universidad Nacional de Córdoba}
\date{Noviembre de 2025}


\begin{document}
	
\begin{frame}[noframenumbering,plain]
    \maketitle
\end{frame}

\section{El Problema}

\begin{frame}{La Ecuación Diferencial}
    El valor de una opción, $V(S, t)$, satisface la Ecuación Diferencial Parcial de Black-Scholes:
    \begin{equation*}
        \frac{\partial V}{\partial t} + \frac{1}{2}\sigma^2 S^2 \frac{\partial^2 V}{\partial S^2} + (r-q)S \frac{\partial V}{\partial S} - rV = 0
    \end{equation*}
    
    \uncover<2->{
    Para resolverla numéricamente: \textbf{Diferencias Finitas} \(\to\) Discretizar el dominio y aproximar las derivadas:
    \[
        \frac{\partial V}{\partial S} \approx \frac{V_{i+1} - V_{i-1}}{2\Delta S}, \quad
        \frac{\partial^2 V}{\partial S^2} \approx \frac{V_{i+1} - 2V_i + V_{i-1}}{(\Delta S)^2}
    \]
    }
    
    \uncover<3->{Pero, ¿Es la misma para las europeas que para las americanas?}
\end{frame}

\begin{frame}[fragile]{Opciones Americanas: Un Problema Complejo}
    Para una opción americana, la EDP se convierte en un \textbf{Problema de Complementariedad Lineal (LCP)}, donde en cada punto del dominio se debe cumplir una de dos condiciones:
    \begin{columns}[T] % Alinea las columnas por la parte superior
        \begin{column}{.5\textwidth}
            \begin{enumerate}
                \item \textbf{Ejercer:} El valor de la opción es su pago intrínseco.
                \[ V(S,t) = \text{Payoff}(S) \]
            \end{enumerate}
        \end{column}
        \begin{column}{.5\textwidth}
            \begin{enumerate}
                \setcounter{enumi}{1}
                \item \textbf{Continuar:} El valor lo da la EDP de Black-Scholes.
                \[ \frac{\partial V}{\partial t} + \mathcal{L}V = 0 \]
            \end{enumerate}
        \end{column}
    \end{columns}
    
    \uncover<2->{
    \vspace{0.5cm}
    Resolver este sistema directamente es complejo.
    \vspace{0.5cm}
    
    \textit{Nuestra estrategia:} Resolver la EDP para la opción europea y luego aplicar un ajuste para incorporar la posibilidad de ejercicio temprano.
    }
\end{frame}

\begin{frame}{Esquema Explícito: Solución Directa}
    \uncover<1->{
    Discretizamos la EDP reemplazando las derivadas por diferencias finitas.
    \vspace{0.5cm}
    }
    
    \uncover<2->{
    Al evaluar las derivadas espaciales en el tiempo ya conocido (\(j-1\)), podemos despejar el valor de la opción en el siguiente paso (\(j\)) de forma directa:
    \[
        V_{i,j} = A'_i V_{i-1,j-1} + B'_i V_{i,j-1} + C'_i V_{i+1,j-1}
    \]
    Donde \(A'_i, B'_i, C'_i\) son coeficientes que dependen de los parámetros del modelo.
    \vspace{0.5cm}
    }
    
    \uncover<3->{
    Para la opción \textbf{americana}, simplemente aplicamos la condición de ejercicio al final de cada paso:
    \[
        V_{i,j}^{\text{final}} = \max(V_{i,j}, \text{Payoff}_i)
    \]
    }
    
\end{frame}

\begin{frame}{Esquema Implícito: Un Sistema de Ecuaciones}
    \uncover<1->{
    Aquí, las derivadas espaciales se evalúan en el tiempo \textbf{desconocido} (\(j\)).
    \vspace{0.5cm}
    }
    
    \uncover<2->{
    Esto nos lleva a un \textbf{sistema de ecuaciones lineales tridiagonal} en cada paso:
    \[
        A_i V_{i-1,j} + B_i V_{i,j} + C_i V_{i+1,j} = V_{i, j-1}
    \]
    }
    
    \uncover<3->{
    Para opciones americanas, el método de \textbf{proyección} consiste en:
    \begin{enumerate}
        \item Resolver el sistema lineal para obtener un valor de continuación \(V_j^{\text{hold}}\).
        \item Aplicar la condición de ejercicio: \(V_j^{\text{final}} = \max(V_j^{\text{hold}}, \text{Payoff})\).
    \end{enumerate}
    }
    
    \uncover<4->{
    \begin{alertblock}{Inconsistencia Teórica}
        Aunque en la práctica funciona bien, la solución final ya no satisface el sistema lineal original.
    \end{alertblock}
    }
    
    \uncover<5->{
    \pause
    Para resolver esto, se usan métodos más sofisticados como \textbf{PSOR (Projected Successive Over-Relaxation)}, que incorporan la restricción en cada paso de la solución iterativa.
    }
\end{frame}


\begin{frame}{Comparación de Métodos}
    \begin{table}
        \centering
        \small
        \begin{tabular}{lll}
            \toprule
            & \textbf{Explícito} & \textbf{Implícito} \\
            \midrule
            \textbf{Simplicidad} & $\checkmark$ Muy simple & Requiere sistema \\
            \textbf{Velocidad} & $\checkmark$ Muy rápido & Más lento \\
            \textbf{Estabilidad} & Condicionada & $\checkmark$ Incond. estable \\
            \bottomrule
        \end{tabular}
    \end{table}
    
    \vspace{0.3cm}
    
\end{frame}

\section{Resultados}

\begin{frame}{Comparación de Precios}
    \begin{figure}
        \centering
        \includegraphics[width=0.95\textwidth]{imagenes/precio_vs_C.png}
    \end{figure}
\end{frame}

\begin{frame}{Comparación de Errores}
    \begin{figure}
        \centering
        \includegraphics[width=0.95\textwidth]{imagenes/error_vs_C.png}
    \end{figure}
\end{frame}

\begin{frame}{Comparación de Velocidad}
    \begin{figure}
        \centering
        \includegraphics[width=0.95\textwidth]{imagenes/velocidad_vs_C.png}
    \end{figure}
\end{frame}

\begin{frame}{Condición de Contorno Artificial}
    \begin{figure}
        \centering
        \includegraphics[width=0.95\textwidth]{imagenes/Cambio_en_L.png}
    \end{figure}
\end{frame}

\begin{frame}{Valuando una Opción Real: NVIDIA}
    \begin{figure}
        \centering
        \includegraphics[width=1\textwidth]{imagenes/Opciones_reales.png}
    \end{figure}
\end{frame}

\begin{frame}{Resultados Obtenidos}
    \textbf{Opción Put Americana ATM sobre NVIDIA} (Vencimiento: 15 días)
    
    \vspace{0.5cm}
    
    \begin{table}
        \centering
        \small
        \begin{tabular}{lr}
            \toprule
            \textbf{Parámetro/Resultado} & \textbf{Valor} \\
            \midrule
            Precio del activo (S) & \$189.44 \\
            Strike (K) & \$190.00 \\
            Volatilidad implícita ($\sigma$) & 56.47\% \\
            Tasa libre de riesgo (r) & 3.875\% \\
            \midrule
            Precio de mercado & \$8.80 \\
            Precio BS Europeo & \$8.783 \\
            \textbf{Precio modelo (PSOR)} & \textbf{\$8.799} \\
            \midrule
            Valor ejercicio temprano & \$0.017 \\
            \bottomrule
        \end{tabular}
    \end{table}
        
\end{frame}

\section{Conclusion}

\begin{frame}{Conclusion}
    \begin{itemize}
        \item Item 1
        \item Item 2
    \end{itemize}	
\end{frame}

	
\end{document}
