% Informe del proyecto (≤ 3 páginas, 11pt, A4)
\documentclass[11pt,a4paper]{article}

% Idioma y codificación
\usepackage[spanish, es-noquoting]{babel}
\usepackage[utf8]{inputenc}
\usepackage[T1]{fontenc}
\usepackage{lmodern}

% % Matemática y figuras
\usepackage{amsmath, amssymb}
\usepackage{graphicx}
\usepackage{booktabs}
\usepackage{microtype}
\usepackage{geometry}
\usepackage{hyperref}
\geometry{margin=2.5cm}
\hypersetup{colorlinks=true,linkcolor=blue,urlcolor=blue,citecolor=blue}

% Datos del trabajo
\title{Diferencias finitas para valorar opciones americanas\\con cambio de variable $Z \to \ln S$}
\author{Mikael Rodriguez \\ Tomás Gómez Pizarro}
\date{Noviembre de 2025}

\begin{document}
\maketitle

\section{Introducción}

\section{Presentación del problema}

El valor de una opción, $V(S, t)$, sobre un activo que paga un dividendo continuo a una tasa $q$, satisface la ecuación diferencial parcial de Black-Scholes:
\[
\frac{\partial V}{\partial t} + \frac{1}{2}\sigma^2 S^2 \frac{\partial^2 V}{\partial S^2} + (r-q)S \frac{\partial V}{\partial S} - rV = 0
\]
Esta ecuación es la misma para opciones de compra (call) y de venta (put), pero sus condiciones de contorno y finales son diferentes.

Para una opción de compra americana sobre un activo que no paga dividendos ($q=0$), nunca es óptimo ejercerla antes del vencimiento. Por lo tanto, su valor es el mismo que el de una opción europea, el cual puede calcularse directamente con la fórmula de Black-Scholes.

En cambio, para una opción de venta americana (incluso sin dividendos) o una opción de compra americana sobre un activo que sí paga dividendos, el ejercicio temprano puede ser óptimo. Esto hace que su valoración requiera métodos numéricos.

\subsection{El cambio de variable}

\subsection{Condiciones de borde}

\subsection{El método explicito vs implicito}

\subsection{Nota sobre la simplificación del método implícito}

\section{Resultados obtenidos}

\subsection{Velocidades y precisión en los distintos métodos}

Se comparó los distintos métodos abordados (explícito, implícito con proyección y implícito con PSOR) en cuestión de precisión y tiempo de cómputo, variando el parámetro $C$ tq \(\Delta Z =  C \sqrt{\Delta t}\) que controla la relación entre los pasos espaciales y temporales en la discretización. Los gráficos de los resultados obtenidos se pueden ver en \href{https://pde-results-visualizer.streamlit.app/}{pde-results-visualizer.streamlit.app}.

\begin{itemize}
    \item \textbf{Precisión:} Para medir el error, se uso la librería QuantLib, que proporciona una valuación precisa de derivados. El método PSOR implícito tuvo una mejor precisión, aunque leve, que el método de proyección simple, lo cuál era de esperar. Todos los métodos son de órden 1 en tiempo y 2 en el espacio, por lo que el método explicito no obtuvo peores resultados que los métodos implícitos. Sin embargo, divergía para valores grandes de \(C\) (es decir, cuando \(\Delta t\) es muy grande en relación a \((\Delta Z)^2\)).
    
    \item \textbf{Velocidad:} Como era de esperar, el método explícito fue el más rápido, seguido por el implícito con proyección y finalmente el implícito con PSOR.
    
    \item \textbf{Impacto de la discretización:} Reducir \(\Delta Z\) (un \(C\) más pequeño) más allá de cierto punto no mejora sustancialmente la precisión, pero sí incrementa el tiempo de cómputo. Para una mayor precisión, es más eficiente reducir el paso temporal \(\Delta t\). El libro de referencia \cite{Hull} propone la elección \(\Delta Z = \sqrt{3\Delta t}\) (equivalente a \(C=\sqrt{3}\)).
    
    \item \textbf{Valor del ejercicio temprano:} Como es de esperar, el valor de la opción americana siempre resulta superior al de su contraparte europea.
\end{itemize}

\subsection{La región de ejercicio temprano}

\subsection{La condición de contorno artificial}

Para resolver la EDP numéricamente, el dominio infinito del precio del activo se trunca a un intervalo finito. Este se define en la variable logarítmica \(Z = \ln S\) mediante un parámetro \(L\), resultando en el dominio computacional \(Z \in \big[\ln K \pm L\sigma\sqrt{T}\big]\).

La elección de \(L\) implica un compromiso. Como se observa en la Figura~\ref{fig:error_vs_L}, un valor de \(L\) demasiado pequeño introduce un error significativo porque las condiciones de contorno artificiales están demasiado cerca de la región de interés. Por otro lado, un \(L\) demasiado grande (con \(M\) fijo) aumenta el paso de discretización \(\Delta Z\), haciendo que el error de truncamiento domine.

\begin{figure}[h!]
    \centering
    \includegraphics[width=0.9\textwidth]{imagenes/Cambio_en_L.png}
    \caption{Error absoluto en función del parámetro de ancho de dominio \(L\), con \(M=200\) y \(N=5000\) fijos.}
    \label{fig:error_vs_L}
\end{figure}

El análisis empírico revela una región óptima para \(L\) (aproximadamente entre 4 y 8) donde el error total es mínimo. Basado en esto, se eligió un valor de \(L=6\) para los demás experimentos, asegurando que la solución no sea sensible a las fronteras artificiales.
\subsection{Evaluando el modelo con datos reales}

\begin{thebibliography}{9}
\bibitem{Hull}
J. C. Hull, \emph{Options, Futures, and Other Derivatives}, Sección 21.8.
\end{thebibliography}

\end{document}
