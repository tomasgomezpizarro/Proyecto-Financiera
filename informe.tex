% Informe del proyecto (≤ 3 páginas, 11pt, A4)
\documentclass[11pt,a4paper]{article}

% Idioma y codificación
\usepackage[spanish, es-noquoting]{babel}
\usepackage[utf8]{inputenc}
\usepackage[T1]{fontenc}
\usepackage{lmodern}

% % Matemática y figuras
\usepackage{amsmath, amssymb}
\usepackage{graphicx}
\usepackage{booktabs}
\usepackage{microtype}
\usepackage{geometry}
\usepackage{hyperref}
\geometry{margin=2.5cm}
\hypersetup{colorlinks=true,linkcolor=blue,urlcolor=blue,citecolor=blue}

% Datos del trabajo
\title{Diferencias finitas para valorar opciones americanas\\con cambio de variable $Z \to \ln S$}
\author{Mikael Rodriguez \\ Tomás Gómez Pizarro}
\date{Noviembre de 2025}

\begin{document}
\maketitle

\section{Introducción}

\section{Presentación del problema}

El valor de una opción, $V(S, t)$, sobre un activo que paga un dividendo continuo a una tasa $q$, satisface la ecuación diferencial parcial de Black-Scholes:
\[
\frac{\partial V}{\partial t} + \frac{1}{2}\sigma^2 S^2 \frac{\partial^2 V}{\partial S^2} + (r-q)S \frac{\partial V}{\partial S} - rV = 0
\]
Esta ecuación es la misma para opciones de compra (call) y de venta (put), pero sus condiciones de contorno y finales son diferentes.

Para una opción de compra americana sobre un activo que no paga dividendos ($q=0$), nunca es óptimo ejercerla antes del vencimiento. Por lo tanto, su valor es el mismo que el de una opción europea, el cual puede calcularse directamente con la fórmula de Black-Scholes.

En cambio, para una opción de venta americana (incluso sin dividendos) o una opción de compra americana sobre un activo que sí paga dividendos, el ejercicio temprano puede ser óptimo. Esto hace que su valoración requiera métodos numéricos.

\subsection{El cambio de variable}

\subsection{Condiciones de borde}

\subsection{El método explicito vs implicito}

\subsection{Nota sobre la simplificación del método implícito}

\section{Resultados obtenidos}

\subsection{Velocidades y precisión en los distintos métodos}

\subsection{La región de ejercicio temprano}

\subsection{La condición de contorno artificial}

Para resolver la EDP numéricamente, es necesario truncar el dominio infinito del precio del activo, \(S \in [0, \infty)\), a un intervalo finito. En nuestro enfoque, este truncamiento se realiza en la variable logarítmica \(Z = \ln S\) y es controlado por un parámetro \(L\). El dominio computacional se define como:
\[
Z \in \big[\ln K - L\sigma\sqrt{T}, \ln K + L\sigma\sqrt{T}\big]
\]
Esto es equivalente a un dominio en precios de \(S \in \big[K e^{-L\sigma\sqrt{T}}, K e^{L\sigma\sqrt{T}}\big]\). En estas fronteras artificiales, se imponen condiciones de contorno que asumen un comportamiento conocido de la opción (por ejemplo, para una put, que su valor se aproxima a \(K\) para \(S \to 0\) y a \(0\) para \(S \to \infty\)).

La elección de \(L\) introduce un compromiso fundamental en la precisión del método. Si se mantienen fijos los pasos de discretización espacial (\(M\)) y temporal (\(N\)), se observa un comportamiento característico del error, como se ilustra en la Figura~\ref{fig:error_vs_L}.

\begin{figure}[h!]
    \centering
    \includegraphics[width=0.9\textwidth]{imagenes/Cambio_en_L.png}
    \caption{Error absoluto en el precio de la opción en función del parámetro de ancho de dominio \(L\), manteniendo fijos el número de pasos espaciales (\(M=200\)) y temporales (\(N=5000\)).}
    \label{fig:error_vs_L}
\end{figure}

Del análisis de la figura se desprenden dos conclusiones clave:
\begin{itemize}
    \item \textbf{\(L\) demasiado pequeño}: Para valores bajos de \(L\) (e.g., \(L < 4\)), el error es significativamente alto. Esto ocurre porque el dominio es demasiado estrecho y la solución numérica es "contaminada" por las condiciones de contorno artificiales, que no representan adecuadamente el comportamiento real de la opción lejos del strike.
    \item \textbf{\(L\) demasiado grande}: A medida que \(L\) aumenta más allá de un punto óptimo (e.g., \(L > 8\)), el error comienza a crecer nuevamente. Con \(M\) fijo, un dominio más grande implica un paso de discretización espacial \(\Delta Z = \frac{2L\sigma\sqrt{T}}{M}\) mayor. Por lo tanto, el error de discretización, que depende de \(\Delta Z\), se vuelve la fuente dominante de imprecisión.
\end{itemize}

El gráfico muestra que existe una región óptima para \(L\) (aproximadamente entre 4 y 8) donde el error combinado de las condiciones de contorno y la discretización es mínimo. Esto justifica la elección de un valor como \(L=6\) para los otros experimentos, ya que se encuentra en esta zona de estabilidad.

\subsection{Evaluando el modelo con datos reales}

\begin{thebibliography}{9}
\bibitem{Hull}
J. C. Hull, \emph{Options, Futures, and Other Derivatives}, Sección 21.8.
\end{thebibliography}

\end{document}
